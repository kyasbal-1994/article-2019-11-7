\documentclass[a4]{article}
%\usepackage{html,htmllist}
%\usepackage{lgrind}
%\usepackage[dvips]{graphicx,color}
\usepackage[dvipdfmx]{graphicx,color}
\usepackage{amsmath}
\usepackage{amsthm}
\usepackage{minted}
\usemintedstyle{pastie}

\newminted[sh]{sh}{fontfamily=tt,fontshape=\upshape,frame=lines,framesep=2mm}

\topmargin=-2.0cm
\textheight=25.0cm
\textwidth=17.0cm
\oddsidemargin=-1.0cm
\evensidemargin=-1.0cm
\renewcommand{\baselinestretch}{0.95}%ページ数が伸びたときにはここに
                                %1.0以下の数字を入れてみましょう
\title{Progress report}
%\author{B4 鄭 中翔} %名前、学年を入れましょう
\author{M2 Kakeru Ishii} %名前、学年を入れましょう

\begin{document}
\twocolumn{
\maketitle
}

\section{Rewrote all code depending JOGL to depend LWJGL}

Recently I had worked to make my renderer working in headless environment.
To make it works, my renderer can't create OpenGL context in straight forward way.
Basic linux OpenGL context is typically strongly bound to X window system.
It may make it harder for my renderer to work in server environment like TSUBAME.

My renderer should use EGL to create headless OpenGL context for this usage.
I have used JOGL(OpenGL java wrapper) in my renderer code, but it was very hard to use EGL over this library.
I switched base library JOGL to LWJGL.

\section{Multi process experiment in TSUBAME}

Most of multi process experiments in TSUBAME might use MPI or the other common parallel computing framework.
My renderer need to be managed manually because I don't use these such kind of frameworks.


As an introduction for TSUBAME job, I need to write batch script(Figure \ref{code: batch}) and execution code(Figure \ref{code: exec}).
The generated file located in environment variable \$PE\_HOSTFILE includes hostnames of slave nodes.
Then qrsh command perform a sort of ssh operation for slave nodes.

\begin{figure*}
    \begin{minted}[
        frame=lines,
        linenos
    ]{sh}
#!/bin/sh
#$ -cwd
# Node type and count for this job
#$ -l s_gpu=2
# Max time for this job
#$ -l h_rt=00:30:00
#$ -V

for n in $(awk '{print $1}' $PE_HOSTFILE)
do
    qrsh -inherit $n /path/to/program &
done
wait    
    \end{minted}
    \caption{Batch job code for TSUBAME to invoke multi process program manually}
    \label{code: batch}
\end{figure*}

\begin{figure*}
    \begin{minted}[
        frame=lines,
        linenos
    ]{sh}
qsub -g <Group name in tsubame> task.sh
    \end{minted}
    \caption{Shell code to put batch job in TSUBAME}
    \label{code: exec}
\end{figure*}

\section{Sync multiple server processes}

I need to run 2 type of nodes on TSUBAME cluster. 
Then I need to sync server processes because merge node should run after running content node.
And each node should know all of host names they need to connect.

I will make use of shared mounted file system for this. 
Each node will write their hostname to specific file to notify their hostname to the other nodes.

\begin{figure}[H]
\begin{enumerate}
    \item Create a file and write own hostname to shared directory
    \item Wait all nodes to complete 1st step
    \item Execute all of content nodes with arguments specified by node index
    \item Wait all content node to be started
    \item Execute all merge nodes
\end{enumerate}
\end{figure}

%\bibliographystyle{acmsiggraph}
\bibliographystyle{unsrt}
\nocite{*}
\small
\bibliography{references} %参考文献データの入ったファイル名をここに書
                          %きます。

\end{document}
